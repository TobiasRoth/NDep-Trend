%% Submissions for peer-review must enable line-numbering
%% using the lineno option in the \documentclass command.
%%
%% Preprints and camera-ready submissions do not need
%% line numbers, and should have this option removed.
%%
%% Please note that the line numbering option requires
%% version 1.1 or newer of the wlpeerj.cls file, and
%% the corresponding author info requires v1.2

\documentclass[fleqn,10pt,lineno]{wlpeerj} % for journal submissions

% ZNK -- Adding headers for pandoc

\setlength{\emergencystretch}{3em}
\providecommand{\tightlist}{
\setlength{\itemsep}{0pt}\setlength{\parskip}{0pt}}
\usepackage{lipsum}
\usepackage[unicode=true]{hyperref}
\usepackage{longtable}



\usepackage{lipsum}

\title{Recovery of mountain plant communities in response to reductions in
Nitrogen emissions is hidden by other drivers of global change}

\author[1, 2]{Tobias Roth}

\corrauthor[1, 2]{Tobias Roth}{\href{mailto:t.roth@unibas.ch}{\nolinkurl{t.roth@unibas.ch}}}
\author[2]{Lukas Kohli}


\affil[1]{Zoological Institute, University of Basel, Basel, Switzerland}
\affil[2]{Hintermann Weber AG, Austrasse 2a, 4153 Reinach, Switzerland}


%
% \author[1]{First Author}
% \author[2]{Second Author}
% \affil[1]{Address of first author}
% \affil[2]{Address of second author}
% \corrauthor[1]{First Author}{f.author@email.com}

% 

\begin{abstract}
Nitrogen (N) deposition is a major threat to biodiversity of many
habitats. The recent introduction of cleaner technologies in Switzerland
has let to reductions in the emissions of nitrogen oxides, with
affiliated decrease in Nitrogen deposition. We infered different drivers
of community change (i.e.~Nitrogen deposition, climate warming, land-use
change) in Swiss mountain hay meadows. The data were obtained from the
Swiss biodiversity monitoring.
% Dummy abstract text. Dummy abstract text. Dummy abstract text. Dummy abstract text. Dummy abstract text. Dummy abstract text. Dummy abstract text. Dummy abstract text. Dummy abstract text. Dummy abstract text. Dummy abstract text.
\end{abstract}

\usepackage{amsthm}
\newtheorem{theorem}{Theorem}[section]
\newtheorem{lemma}{Lemma}[section]
\theoremstyle{definition}
\newtheorem{definition}{Definition}[section]
\newtheorem{corollary}{Corollary}[section]
\newtheorem{proposition}{Proposition}[section]
\theoremstyle{definition}
\newtheorem{example}{Example}[section]
\theoremstyle{definition}
\newtheorem{exercise}{Exercise}[section]
\theoremstyle{remark}
\newtheorem*{remark}{Remark}
\newtheorem*{solution}{Solution}
\begin{document}

\flushbottom
\maketitle
\thispagestyle{empty}

\section*{Introduction}\label{introduction}
\addcontentsline{toc}{section}{Introduction}

Nitrogen (N) deposition is a major threat to biodiversity. The recent
introduction of cleaner technologies has let to reductions in the
emissions of nitrogen oxides, with affiliated decrease in N-deposition
in many parts of Europe. However, it is an open question whether and how
fast the reduction in N deposition rates will lead to the recovey of
extant plant communities.

One useful approach to understanding biodiversity change is through
estimates of biodiversity turnover reflecting both immigration and
extinction, often in a closed range of values
(Hillebrand\_et\_al-2018.pdf).

Here we infered mountain hay meadows in Switzerland. Explain why
mountain hay meadows are important. Also explain other threats to
mountain hay meadows (climate change, land-use change).

\section*{Materials \& Methods}\label{materials-methods}
\addcontentsline{toc}{section}{Materials \& Methods}

\subsection*{Monitoring data and community
measures}\label{monitoring-data-and-community-measures}
\addcontentsline{toc}{subsection}{Monitoring data and community
measures}

We analysed the presence/absence of vascular plants sampled within the
scope of Switzerland's Biodiversity Monitoring (BDM) programme that was
launched in 2001 to monitor Switzerland's biodiversity and to comply
with the Convention on Biological Diversity of Rio de Janeiro (Weber,
Hintermann, and Zangger 2004). The sampling sites were circles with a
size of 10 m\(^2\) and data collection was carried out by qualified
botanists who visited each sampling site twice within the same season.
During each visit all the vascular plant species detected on the plot
were recorded. After the sampling of the plant data the botanists also
assigned a habitat type to each sampling site according to the
classification system developed for Switzerland (Delarze and Gonseth
2008). For more details on the field methods see Plattner, Birrer, and
Weber (2004), Roth et al. (2013) and Roth et al. (2017).

We matched the habitat types of the Swiss classification system with the
categories from the EUNIS system (level-3 classification; Davies, Moss,
and Hill 2004) and selected all sampling sites in mountain hay meadows
(EUNIS E2.3). We analysed the data from 2003 to 2017. During that study
period each sampling site was surveyed once per five-year period: the
first period lasted from 2003 to 2007, the second from 2008 to 2012 and
the third from 2013 to 2017. These selection criterias resulted in
sample of 129 sites.

For each survey (a survey conists of the two visits within the same
season) we calculated the following community measures: (1) The number
of recorded species (species richness), (2) the community mean of the
Ellenberg temperature values of recorded species (community temperature
index), (3) the community mean of the Ellenberg humidity values of
recorded species (community humidity index), (4) the community mean of
the Ellenberg nutrients values of recorded species (community nutrients
index) and (5) the community mean of the Ellenberg light values of
recorded species (community light index). The Ellenberg values were
obtained from the recalibrated indicator values for the Swiss Flora
(Landolt et al. 2010). Additionally to these five community measures
that describe the state of plant communities for each site at a given
time point, we also estimated the temporal turnover (i.e.~species
exchange ratio sensu Hillebrand et al. (2018)) as the proportion of
species that differ between two time points to describe the community
change between two time points.

\subsection*{Environmental gradients}\label{environmental-gradients}
\addcontentsline{toc}{subsection}{Environmental gradients}

\begin{itemize}
\tightlist
\item
  Mean annual temperature
\item
  Mean annual precipitation
\item
  Nitrogen deposition
\item
  Inclination: The steaper the site the less intensive a site.
\item
  Environmental variables were standardized.
\end{itemize}

\subsection*{Statistical analyses}\label{statistical-analyses}
\addcontentsline{toc}{subsection}{Statistical analyses}

To estimate the linear trend over time for each of the five community
measures we applied linear mixed models with normal distribution (except
for alpha-diversity with Poisson distribution and the logoraithm as link
function). We used the year as predictor variable and site-ID as random
intercept and slope variable to account for the fact that every site was
sampled three times. Model parameters were estimated in a Bayesian
framework using the R-Package \emph{rstanarm} (Stan Development Team
2016, Muth, Oravecz, and Gabry (2018)).

\begin{itemize}
\tightlist
\item
  Comparision of colonizating and disappearing species with randomly
  selected species (refer to Appendix A).
\end{itemize}

To estimate the effect of N deposition at a given time point . We
described the plant species richness at the sites using a generalized
linear model with Poisson distribution and the logarihm as link
function. The selection of predictor variables was based on an earlier
study in Swiss mountain grasslands (Roth et al. 2013). Except for N
deposition we thus used elevation (linear and squared term), inclination
(linear and squared term), mean annual precipitation (linear and squared
term), pH, average moisture value, average light value and the aspect.
For the analyses the variables were standardized.

\section*{Results}\label{results}
\addcontentsline{toc}{section}{Results}

\subsection*{Temporal change in community
structures}\label{temporal-change-in-community-structures}
\addcontentsline{toc}{subsection}{Temporal change in community
structures}

\begin{table}

\caption{\label{tab:communitytrendstab}Average measures of community structure for the three sampling periods (period 1: 2003-2007; period 2: 2008-2012; period: 2013-2017). The temporal trends are given as change per 10 years and were estimated from linear mixed models with normal distribution (except for alpha-diversity with Poisson distribution and a log-link function). The measure of precision for the temporal trend is given as the 5\% and 95\% quantiles of the marginal posterior distribution of the linear trend. The column 'Prob. for trend' gives the probability that the linear trend is $> 0$.}
\centering
\begin{tabular}[t]{lrrrrrrr}
\toprule
Measures & Period 1 & Period 2 & Period 3 & Trend & 5\% & 95\% & Prob. for trend\\
\midrule
Alpha-diversity & 45.72 & 46.02 & 45.74 & 0.00 & -0.03 & 0.03 & 0.53\\
Temperature value & 3.11 & 3.13 & 3.13 & 0.01 & 0.00 & 0.03 & 0.97\\
Huminity value & 2.99 & 2.98 & 2.99 & 0.01 & -0.01 & 0.02 & 0.80\\
Nutrients value & 3.20 & 3.20 & 3.20 & 0.00 & -0.02 & 0.01 & 0.33\\
Light value & 3.56 & 3.55 & 3.55 & -0.01 & -0.02 & 0.00 & 0.07\\
\bottomrule
\end{tabular}
\end{table}

The five measures of plant community structure suggested that plant
communities in mountain hay meadows were rather stable between 2003 and
2017 and did not show a clear increase or decrease over time (Table
\ref{tab:communitytrendstab}): for each of the three 5-year survey
periods the averages of alpha-diversity and the average Ellenberg values
for temperature, huminity, nutrients and light did not vary much among
the three sampling periods and the estimated trends were rather small.
Except for average Ellenberg value for temperature, the 90\%
credible-interval of the temporal trend contained zero. The results from
the linear mixed models suggest that a linear temporal change was most
likely for the community mean of the Ellenberg value for temperature
(probability of increase: 0.97), followed by the community mean of the
Ellenberg light value (probability of decrease: 0.93) and it was least
likely for the alpha-diversity (probability of increase: 0.53). The
chance that the community mean of the nutrient value decreased between
2003 and 2017 was 0.67.

\subsection*{Different drivers of species
turnover}\label{different-drivers-of-species-turnover}
\addcontentsline{toc}{subsection}{Different drivers of species turnover}

\begin{table}

\caption{\label{tab:turnovertab}Change of species turnover along the four gradients. The slopes along the gradients (estimate) are given as the change per 10 years of the logit-probability of species that differed between two surveys. Estimates and the 5\% and 95\% quantiles of the marginal posterior distribution obtained from a Binomial-GLMM with the proportion of species that differed between two surveys as dependent variable and the site gradients and period as predictors and site-ID as random effect.}
\centering
\begin{tabular}[t]{lrrr}
\toprule
Gradient & Estimate & 5\% & 95\%\\
\midrule
Annual mean temperature & 0.06 & 0.00 & 0.11\\
Annual mean precipitation & -0.02 & -0.11 & 0.08\\
Nitrogen deposition & -0.20 & -0.35 & -0.05\\
Inclination & -0.04 & -0.10 & 0.03\\
\bottomrule
\end{tabular}
\end{table}

This temporal stability as inferred from the community measures was,
however, in contrast to a rather large observed temporal turnover of
species. The average percentage \(\pm\) SD of species that differed
between the first and second survey at a site was NaN \(\pm\) NA\% and
the percentage of species that differ between the second and third
survey was NaN \(\pm\) NA\%. Thus, it seemed that the turnover from the
first/second survey to the turnover of the second/third survey
moderately decreased (90\% Credible interval of the change in turnover
estimated from the Binomial generalized linear mixed model: -0.15 -
-0.02). Variation in species turnover was largest along the Nitrogen
deposition gradient with highest species turnover at sites with low
Nitrogen deposition (Table \ref{tab:turnovertab}). The other three
gradients were less inportant to explain the variation in species
turnover among sites.

\begin{table}

\caption{\label{tab:difffromrandomtab}Difference in the average Ellenberg value of species that (a) disappeared from site or (b) newly colonized a site compared to the same number of species that were randomly selected from all species recorded at a site. Shown are the results from linear model with the difference between disappeard/colonized species and random species as dependend variable and the sitemeasure (gradient) as predictor variable.}
\centering
\begin{tabular}[t]{llcccccc}
\toprule
\multicolumn{2}{c}{ } & \multicolumn{3}{c}{Difference from random} & \multicolumn{3}{c}{Change along gradient} \\
\cmidrule(l{2pt}r{2pt}){3-5} \cmidrule(l{2pt}r{2pt}){6-8}
Ellenberg value & Gradient & Estimate & 5\% & 90\% & Estimate & 5\% & 90\%\\
\midrule
\addlinespace[0.3em]
\multicolumn{8}{l}{\textit{(a) Plants that disappeard from a site}}\\
\hspace{1em}Temperature & Annual mean temperature & -0.012 & -0.034 & 0.009 & 0.007 & -0.002 & 0.016\\
\hspace{1em}Humidity & Annual mean precipitation & -0.003 & -0.038 & 0.033 & 0.008 & -0.012 & 0.029\\
\hspace{1em}Nutrients & Nitrogen deposition & -0.012 & -0.052 & 0.028 & 0.000 & -0.046 & 0.046\\
\hspace{1em}Light & Inclination & -0.022 & -0.047 & 0.004 & -0.002 & -0.024 & 0.022\\
\addlinespace[0.3em]
\multicolumn{8}{l}{\textit{(b) Plants that newly colonized a site}}\\
\hspace{1em}Temperature & Annual mean temperature & 0.018 & 0.001 & 0.034 & -0.001 & -0.008 & 0.006\\
\hspace{1em}Humidity & Annual mean precipitation & 0.023 & -0.010 & 0.054 & -0.004 & -0.023 & 0.014\\
\hspace{1em}Nutrients & Nitrogen deposition & -0.082 & -0.117 & -0.048 & 0.062 & 0.024 & 0.102\\
\hspace{1em}Light & Inclination & -0.039 & -0.062 & -0.016 & 0.010 & -0.009 & 0.031\\
\bottomrule
\end{tabular}
\end{table}

High species turnover at a site is the result of species that
disappeared from the site and species that newly colonized the site. To
better understand the factors that drive these changes we are
particularly interested whether the species that disappeared or
colonized the sites differed in Ellenberg values compared to what would
be expected if the same number of species randomly disappeared or
colonized the sites (i.e.~random disappearance and random colonization)
and whether there is a change along the gradients. It seems that the
Ellenberg values of newly colonizing species differened more from random
colonization than the Ellenberg values of disappearing species (Table
\ref{tab:difffromrandomtab}). For colonizing species, we found the
largest differences from random colonization in the Ellenberg value for
nutrients: at sites with nitrogen deposition of 10 kg ha\(^{-1}\)
yr\(^{-1}\) the newly colonizing species had in average a lower
Ellenberg value for nutrients than species under random disappearance
(column ``Difference from random'' in Table
\ref{tab:difffromrandomtab}), but this differences between colonizing
species and random colonization decreased with increasing N deposition
(column ``Change along gradient'' in Table \ref{tab:difffromrandomtab}).
Thus at high Nitrogen deposition colonizing species did not differ from
random species (see Figure 3b in Appendix A).

While colonizing species had higher temperature values compared to what
we would expect under random colonization, the differences between
colonizing species and random colonization was about four times smaller
compared to the difference in Ellenberg value for nutrients between
colonizing species and random species. Nevertheless, the variation in
Ellenberg value for temperature seemed impportant to explain the total
species turnover. This is because, disappearing species tend to have
lower temperature value than random species as well as colonizing
species tend to have hihger temperature values than random species; both
processes lead to an overall replacement of species with lower
temperature value with species with higher temperature values. This was
not the case for the Ellenberg value for nutrients: species with lower
nutrients values tended to be more likely to disappear from as well as
to colonize sites compared to random species (Table
\ref{tab:difffromrandomtab}). See also Appendix A were we present
detailed results for the comparision between colonizing or disappearing
species with randomly selected species.

\subsection*{Potential effects of reduction in Nitrogen
emmissions}\label{potential-effects-of-reduction-in-nitrogen-emmissions}
\addcontentsline{toc}{subsection}{Potential effects of reduction in
Nitrogen emmissions}

\begin{figure}
\includegraphics[width=1\linewidth]{Manuscript_files/figure-latex/cl-1} \caption{Colonization (a) and local survival (b) of oligotrophic (Ellenberg N = 2; red line) and eutrophic (Ellengerg N = 4) species along the N deposition gradient. Given are means and 95\%-Credible Intervals from logistic linear mixed models. The vertical lines indicat the deposition rate with equal colonization or survival probabilities for oligotrophic and eutrophic species with the solid line indicating the median and the dashed lines the 5\% and 95\% quantiels of the margional posterior distribution.}\label{fig:cl}
\end{figure}

In Fig. \ref{fig:cl} we compare the colonization and local survival
probability of oligotrophic (Ellenberg value of nutrients = 2) and
eutrophic (Ellenberg value of nutrients = 4) species along the Nitrogen
deposition gradient. Local survival probability was the same for
oligotrophic and euthrophic species at a deposition rate of 11.89 kg N
ha\(^{-1}\) yr\(^{-1}\); colonization probability was the same for for
oligotrophic and euthrophic species at a deposition rate of 12.15 kg N
ha\(^{-1}\) yr\(^{-1}\). In only 0.30\% of the sites the deposition rate
was below 11.5 kg N ha\(^{-1}\) yr\(^{-1}\) where the replacement of
eutrophic with oligotrophic species is likely.

While we could not detect a consistent decrease in average Ellenberg
value for nutrients (Table \ref{tab:communitytrendstab}), the higher
colonization rate of species with low nutrient value at sites with low
deposition rate seems to affect the spatial variation of species
richness: sites with low Nitrogen deposition are likely to become more
species rich over time likely resulting in steeper slope of the negative
relationship between N deposition and species richness. Indeed, if we
apply at different time points a similar model as in Roth et al. (2013)
to infer the effects of N deposition on the spatial variation of species
richness, the resulting effect size (i.e.~the slope) becomes more
negative over time (Fig. \ref{fig:figconsequences}).

\begin{figure}
\includegraphics[width=0.5\linewidth]{Manuscript_files/figure-latex/figconsequences-1} \caption{Effect size of Nitrogen deposition on total species richness estimated from applying the Poisson-GLM with species richness as dependend variable and Nitrogen deposition plus other site covariates as predictors using only the surveys from one five-year interval. Note that within every five-year interval all plots were sampled once.}\label{fig:figconsequences}
\end{figure}

\section*{Discussion}\label{discussion}
\addcontentsline{toc}{section}{Discussion}

\begin{itemize}
\item
  \emph{General points}: Altough N deposition considerabely declinded
  between 2005 and 2015, we could not detect major shifts in plant
  community structure during the same time period.
\item
  \emph{Replacement of oligotrophic with eutrophic species is faster
  than the oposite direction}: Eutrophic species have rather high local
  survival across the entire deposition gradient, while oligotrophic
  species have much reduced local survival at high N deposition. This
  suggests that it takes more time to replace eutrophic by oligotrophic
  species than replacing oligotrophic by eutrophic species. Climatic
  effects may be more likely to be reversed thant effects due to
  fertilization.
\item
  \emph{Methodological point}: The rather large spatial turnover might
  be partly explained by species that remained undetected in one of the
  surveys. However, our results suggest that turnover is caused at least
  partly by species with specific Ellenberg values. These deviation from
  what we would expect under random species turnover is unlikely to be
  explained by species that remained undetected.
\item
  \emph{Empirical critical loads}: Our data on colonization and local
  survival (i.e.~temporal variation) confirm the empirical critical
  loads that we infered from anlysing spatial co-variation of N
  deposition and species richness.
\item
  \emph{Space for time substitution}: Often observational studies infer
  the change of plant diversity along a gradient of N deposi- tion.
  Thus, they infer how the spatial variation in species richness is
  related to N deposition and assume that this spatial variation in
  species richness arose because over time some areas lost more species
  than others because they chronically experienced higher N deposition.
  Alt- hough there is evidence supporting the use of such a `space for
  time substitution' for detecting the effects of N deposition on plant
  diversity (Stevens et al. 2010), they can not replace stud- ies that
  relate temporal patterns in species with N deposition (De Schrijver et
  al. 2011). While recovery of acidified surface waters has been well
  investigated (De Vries et al. 2015), there are only a limited number
  of studies inferring temporal trends of plant species diversity relat-
  ed to varying amounts of N-deposition. Storkey et al. (2015)
  demonstrated a positive response of biodiversity to reducing N
  addition from either atmospheric pollution or fertilizers in the Park
  Grass Experiment: «The proportion of legumes, species richness and
  diversity increased across the experiment between 1991 and 2012 as
  N-deposition declined». For forest floor vegetation in permanent plots
  across Europe the exceedance of critical loads of N over a peri- od
  from 9 to 42 years had negative effects on the cover of oligotrophic
  plant species, i.e spe- cies that prefer nutrient-poor soils, although
  species richness remained constant (Dirnböck et al. 2014). Another
  example of recovery in eutrophicated habitats gives the recovery of
  species richness in previously fertilized plots (Clark and Tilman
  2008). In this study, the recorded recovery in species richness within
  one or two decade was likely due to the species rich vege- tation
  surrounding the experimental plots, from where immigration was easily
  feasible.
\end{itemize}

\section*{Conclusions}\label{conclusions}
\addcontentsline{toc}{section}{Conclusions}

xxx

\section*{Acknowledgements}\label{acknowledgements}
\addcontentsline{toc}{section}{Acknowledgements}

We thank the dedicated and qualified botanists who conducted fieldwork.
The Swiss Federal Office for the Environment (FOEN) kindly provided
biodiversity monitoring data and topographic data. This work was
supported by the FOEN, the Swiss National Science Foundation (grant no.
31003A\_156294), the Swiss Association Pro Petite Camargue Alsacienne,
the Fondation de bienfaisance Jeanne Lovioz, and the MAVA Foundation.

\section*{References}\label{references}
\addcontentsline{toc}{section}{References}

\hypertarget{refs}{}
\hypertarget{ref-Davies2004}{}
Davies, CE, D Moss, and MO Hill. 2004. ``EUNIS Habitat Classification,
Revised 2004. Report to European Environment Agency, European Topic
Centre on Nature Protection and Biodiversity.''

\hypertarget{ref-Delarze2008}{}
Delarze, Raymond, and Yves Gonseth. 2008. \emph{Lebensräume Der Schweiz:
Ökologie - Gefährdung - Kennarten}. Ott.
\url{https://books.google.ch/books?id=8kFrAAAACAAJ}.

\hypertarget{ref-Hillebrand2018}{}
Hillebrand, Helmut, Bernd Blasius, Elizabeth T. Borer, Jonathan M.
Chase, John A. Downing, Britas Klemens Eriksson, Christopher T.
Filstrup, et al. 2018. ``Biodiversity Change Is Uncoupled from Species
Richness Trends: Consequences for Conservation and Monitoring.''
\emph{Journal of Applied Ecology} 55 (1): 169--84.
doi:\href{https://doi.org/10.1111/1365-2664.12959}{10.1111/1365-2664.12959}.

\hypertarget{ref-Landolt2010}{}
Landolt, Elias, Beat Bäumler, Andreas Erhardt, Otto Hegg, Frank Klötzli,
Walter Lämmler, Michael Nobis, et al. 2010. \emph{Flora Indicativa.
Ecological Inicator Values and Biological Attributes of the Flora of
Switzerland and the Alps.} Haupt Verlag.

\hypertarget{ref-Muth2018}{}
Muth, Chelsea, Zita Oravecz, and Jonah Gabry. 2018. ``User-Friendly
Bayesian Regression Modeling: A Tutorial with Rstanarm and Shinystan.''
\emph{The Quantitative Methods for Psychology} 14 (2): 99--119.

\hypertarget{ref-Plattner2004}{}
Plattner, Matthias, Stefan Birrer, and Darius Weber. 2004. ``Data
Quality in Monitoring Plant Species Richness in Switzerland.''
\emph{Community Ecology} 5 (1): 135--43.

\hypertarget{ref-Roth2013}{}
Roth, Tobias, Lukas Kohli, Beat Rihm, and Beat Achermann. 2013.
``Nitrogen Deposition Is Negatively Related to Species Richness and
Species Composition of Vascular Plants and Bryophytes in Swiss Mountain
Grassland.'' \emph{Agriculture, Ecosystems \& Environment} 178: 121--26.
doi:\href{https://doi.org/10.1016/j.agee.2013.07.002}{10.1016/j.agee.2013.07.002}.

\hypertarget{ref-Roth2017}{}
Roth, Tobias, Lukas Kohli, Beat Rihm, Reto Meier, and Beat Achermann.
2017. ``Using Change-Point Models to Estimate Empirical Critical Loads
for Nitrogen in Mountain Ecosystems.'' \emph{Environmental Pollution}
220: 1480--7.

\hypertarget{ref-Stan2016}{}
Stan Development Team. 2016. ``Rstanarm: Bayesian Applied Regression
Modeling via Stan.'' \url{http://mc-stan.org/}.

\hypertarget{ref-Weber2004}{}
Weber, Darius, Urs Hintermann, and Adrian Zangger. 2004. ``Scale and
Trends in Species Richness: Considerations for Monitoring Biological
Diversity for Political Purposes.'' \emph{Global Ecology and
Biogeography} 13 (2): 97--104.



\end{document}
