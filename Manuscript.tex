%% Submissions for peer-review must enable line-numbering
%% using the lineno option in the \documentclass command.
%%
%% Preprints and camera-ready submissions do not need
%% line numbers, and should have this option removed.
%%
%% Please note that the line numbering option requires
%% version 1.1 or newer of the wlpeerj.cls file, and
%% the corresponding author info requires v1.2

\documentclass[fleqn,10pt,lineno]{wlpeerj} % for journal submissions

% ZNK -- Adding headers for pandoc

\setlength{\emergencystretch}{3em}
\providecommand{\tightlist}{
\setlength{\itemsep}{0pt}\setlength{\parskip}{0pt}}
\usepackage{lipsum}
\usepackage[unicode=true]{hyperref}
\usepackage{longtable}



\usepackage{lipsum}

\title{xxx}

\author[1]{Tobias Roth}

\corrauthor[1]{Tobias Roth}{\href{mailto:t.roth@unibas.ch}{\nolinkurl{t.roth@unibas.ch}}}
\author[2]{Lukas Kohli}


\affil[1]{Zoological Institute, University of Basel, Basel, Switzerland}
\affil[2]{Hintermann Weber AG, Austrasse 2a, 4153 Reinach, Switzerland}


%
% \author[1]{First Author}
% \author[2]{Second Author}
% \affil[1]{Address of first author}
% \affil[2]{Address of second author}
% \corrauthor[1]{First Author}{f.author@email.com}

% 

\begin{abstract}
xxx
% Dummy abstract text. Dummy abstract text. Dummy abstract text. Dummy abstract text. Dummy abstract text. Dummy abstract text. Dummy abstract text. Dummy abstract text. Dummy abstract text. Dummy abstract text. Dummy abstract text.
\end{abstract}

\begin{document}

\flushbottom
\maketitle
\thispagestyle{empty}

\section*{Introduction}\label{introduction}
\addcontentsline{toc}{section}{Introduction}

xxx

\section*{Materials \& Methods}\label{materials-methods}
\addcontentsline{toc}{section}{Materials \& Methods}

\subsection*{Monitoring data}\label{monitoring-data}
\addcontentsline{toc}{subsection}{Monitoring data}

\begin{itemize}
\tightlist
\item
  Selection of sample sites based on 1366 K\_Standort.csv column
  ``E23\_1366''.
\item
  Three surveys 2003-2007, 2008-2012 and 2013 - 2017.
\end{itemize}

\subsection*{Plant traits}\label{plant-traits}
\addcontentsline{toc}{subsection}{Plant traits}

Functional traits:

\begin{itemize}
\tightlist
\item
  SLA: specific leaf area
\item
  CH: canopy height
\item
  SM: Seed mass
\end{itemize}

Ellenberg indicator values:

\begin{itemize}
\tightlist
\item
  L: light
\item
  N: Nutrient contentent
\item
  T: Temperature
\item
  F: Huminity
\end{itemize}

Community measures:

\begin{itemize}
\tightlist
\item
  Species richness: number of recorded species per \(10m^2\).
\item
  Spatial turnover (beta-diversity): Average turnover between all
  pair-wise combinations of study plots.
\item
  gamma diversity: Total number of species recorded in all study plots.
\end{itemize}

\subsection*{Statistical analyses}\label{statistical-analyses}
\addcontentsline{toc}{subsection}{Statistical analyses}

Environmental variables were standardized.

\section*{Results}\label{results}
\addcontentsline{toc}{section}{Results}

Different measures of total community structure suggested that plant
communities of mountain hay meadow were rather stable between 2003 and
2017: For each of the three 5-year survey periods the averages of
alpha-, beta- and gamma-diversity, average ellenberg values for
temperature, nutrients, light, huminity and reaction, and average of
species' canopy height, specific leaf area and seed mass did not vary
much. For all measures, the average temperal trend per site did not
differ significantly from zero (Table \ref{temporaltrends}). Note that
beta- and gamma-diversity are note available for single sites and thus
mixed models could not be applied.

\begin{table}[ht]
\centering
\begin{tabular}{lrrrrr}
  \hline
Measures & Period 1 & Period 2 & Period 3 & Temporal-Trend & P-value \\ 
  \hline
Alpha-diversity & 46.36 & 46.72 & 46.45 & 0.002 & 0.896 \\ 
  Beta-diversity & 0.68 & 0.65 & 0.65 &  &  \\ 
  Gamma-Diversity & 517 & 529 & 517 &  &  \\ 
  Temperature value & 3.12 & 3.14 & 3.14 & 0.013 & 0.060 \\ 
  Huminity value & 2.99 & 2.98 & 2.99 & 0.006 & 0.405 \\ 
  Nutrients value & 3.22 & 3.22 & 3.22 & -0.004 & 0.698 \\ 
  Light value & 3.57 & 3.56 & 3.56 & -0.010 & 0.196 \\ 
  Canopy height & -1.24 & -1.22 & -1.23 & 0.013 & 0.307 \\ 
  Specific leaf area & 8.21 & 8.27 & 8.24 & 0.030 & 0.621 \\ 
  Seed mass & -0.34 & -0.32 & -0.33 & 0.010 & 0.596 \\ 
   \hline
\end{tabular}
\caption{Average measures of community structure for the three survey periods (in each period all sites are surveyed once). The temporal trends and p-values are based on linear mixed models with normal distribution (except for alpha-diversity with Poisson distribution) with site-ID as random effect. Temporal-trends are given per 10 years. Linear mixed models could not be applied for beta- and gamma-diversity because measures are not available for the single sites.} 
\label{temporaltrends}
\end{table}

The temporal stability suggested by the community measures were,
however, in contrast to the temporal turn-over of recorded species
(i.e.~species exchange ratio sensu Hillebrand et al. (2018)). In average
\(\pm\) SD the proportion of species that differed between two surveys
was 18.7 \(\pm\) 6.8.

\begin{verbatim}
## 
## Call:
## lm(formula = Turnover ~ NTOT2007 + Hoehe + Neig + Expos, data = sites)
## 
## Residuals:
##       Min        1Q    Median        3Q       Max 
## -0.148019 -0.040275 -0.007297  0.042223  0.205389 
## 
## Coefficients:
##               Estimate Std. Error t value Pr(>|t|)    
## (Intercept)  3.468e-01  5.191e-02   6.682 6.98e-10 ***
## NTOT2007    -4.629e-03  1.498e-03  -3.091  0.00246 ** 
## Hoehe       -7.279e-05  2.263e-05  -3.216  0.00165 ** 
## Neig        -1.816e-04  7.590e-04  -0.239  0.81125    
## Expos        2.433e-05  2.401e-05   1.014  0.31278    
## ---
## Signif. codes:  0 '***' 0.001 '**' 0.01 '*' 0.05 '.' 0.1 ' ' 1
## 
## Residual standard error: 0.06522 on 125 degrees of freedom
## Multiple R-squared:  0.1085, Adjusted R-squared:  0.07997 
## F-statistic: 3.803 on 4 and 125 DF,  p-value: 0.005939
\end{verbatim}

\section*{Discussion}\label{discussion}
\addcontentsline{toc}{section}{Discussion}

xxx

\section*{Conclusions}\label{conclusions}
\addcontentsline{toc}{section}{Conclusions}

xxx

\section*{Acknowledgements}\label{acknowledgements}
\addcontentsline{toc}{section}{Acknowledgements}

xxx

\section*{References}\label{references}
\addcontentsline{toc}{section}{References}

\hypertarget{refs}{}
\hypertarget{ref-Hillebrand2018}{}
Hillebrand, Helmut, Bernd Blasius, Elizabeth T. Borer, Jonathan M.
Chase, John A. Downing, Britas Klemens Eriksson, Christopher T.
Filstrup, et al. 2018. ``Biodiversity Change Is Uncoupled from Species
Richness Trends: Consequences for Conservation and Monitoring.''
\emph{Journal of Applied Ecology} 55 (1): 169--84.
doi:\href{https://doi.org/10.1111/1365-2664.12959}{10.1111/1365-2664.12959}.



\end{document}
